\documentclass[14pt,extrafontsizes]{article}
\usepackage[a4paper,left=2cm,right=2cm,top=3cm,bottom=3cm]{geometry}
\usepackage{graphicx} % Required for inserting images
\usepackage[russian]{babel}
\usepackage[T1]{fontenc}
\usepackage[linguistics]{forest}
\usepackage{hyperref}
\usepackage{fancyvrb}
\usepackage{colortbl}
\usepackage[table]{xcolor}
\usepackage{xcolor}
\usepackage[dvipsnames]{xcolor}
\title{Передвижения. ДЗ5}
\author{Екатерина Ткаченко 232}

\begin{document}
\maketitle
\section*{Задание 1.}
\begin{enumerate}
    \item \textsc{Из этих заданий [я сделал каждое].}\\
        Присутствие падежной связности
    \item \textsc{This is not what I’ve been looking for.}\\
        Присутствие падежной связности
    \item \textsc{Кому я просила, [чтобы ты помог]}\\
    Присутствие падежной связности
    \item \textsc{Пекари из нас вышли впечатляющие.}\\
    Присутствие оставшихся в начальной позиции зависимых
    \item \textsc{Я выразил выступавшим на конференции лингвистам по благодарности каждому.}\\
    \item \textsc{Лингвист из меня получится непутевый.}\\
        Присутствие оставшихся в начальной позиции зависимых
    \item \textsc{С кем он вам сказал, [что я виделся]?}\\ 
    Присутствие пажежной связности
    \item \textsc{Детей я проверил каждого.}\\
        Присутствие падежной связности
    \item \textsc{What I’ve been looking for was this.}\\
\end{enumerate}
\section*{Задание 2.}
\begin{enumerate}
\item Would you be kind enough to explain to me how head movement works?\\
\begin{forest}
    [CP [C$'$
        [C {[+Q]} [C {[+Q]}] [T [Would, name=T1]]]
        [TP [DP[You]] [T$'$ [t(i), name=T2] [VP [{be kind enough \\ to explain to me \\ how head movement works?},roof ]]]]
    ]]
    \draw[->] (T2)
    to[out=south west, in=south west](T1)
\end{forest}\\
\textbf{ОТВЕТ: Вершинное передвижение}
\item What do you see in this picture?\\
\begin{forest}
    [CP [DP [What, name=T1, roof ]] [C$'$ [C [do]] [TP [DP [You,roof ]][T$'$ [T [\emptyset]] [VP [V$'$[V$'$ [V [see]] [t(i), name=T2]][PrepP [{in this picture?},roof ]]]]]]]]
    \draw[->] (T2)
    to[out=south west, in=south west](T1)
\end{forest}\\
\textbf{ОТВЕТ: Фразовое передвижение}
\item Does John love nanosyntax?\\
\textbf{ОТВЕТ: Вершинное передвижение}
\item The students can skip some lectures.\\
\begin{forest}
    [TP [DP [{The students},name=T1,roof]][T$'$[T[can]][VP[t(i), name=T2][V$'$[V[skip]][DP[{some lectures},roof ]]]]]]
    \draw[->] (T2)
    to[out=south west, in=south west](T1)
\end{forest}\\
\textbf{ОТВЕТ: Вершинное передвижение}
\item Яблоко, растущее на каком из этих двух деревьев, мне надо съесть?\\
\textbf{ОТВЕТ: Фразовое передвижение}
\item The students were read a lecture on generative syntax.\\
\textbf{ОТВЕТ: Вершинное передвижение}
\item Hoffman schreibt oft Musik.\\
\begin{forest}
    [TP [DP [Hoffman,roof ]] [T$'$ [T+V [schreibt, name=T1]][VP [t(i)][V$'$ [AdvP[oft]][V$'$[t(j), name=T2][DP[Musik,roof ]]]]]]]
    \draw[->] (T2)
    to[out=south west, in=south west](T1)
\end{forest}\\
\textbf{ОТВЕТ: Вершинное передвижение}
\item Кривую козу привел Иларион(, а хромую – Амфитрион.)\\
\textbf{ОТВЕТ: Фразовое передвижение}
\end{enumerate}
\end{document}