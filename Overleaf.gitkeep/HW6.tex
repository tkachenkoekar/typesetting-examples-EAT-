\documentclass[16pt]{article}
\usepackage[a4paper,left=2cm,right=2cm,top=3cm,bottom=3cm]{geometry}
\usepackage{graphicx} % Required for inserting images
\usepackage[russian]{babel}
\usepackage[T1]{fontenc}
\usepackage[linguistics]{forest}
\usepackage{hyperref}
\usepackage{fancyvrb}
\usepackage{colortbl}
\usepackage[table]{xcolor}
\usepackage{xcolor}
\usepackage[dvipsnames]{xcolor}
\title{Вокруг GR. ДЗ6}
\author{Екатерина Ткаченко 232}

\begin{document}
\maketitle

\section*{Задание 1.}
Нужно выбрать пример В) про водский, потому что DOM = Differential Object Marking, а тут как раз дифференциально маркируется объект. А DOM здесь нужен для того, чтобы указать на категорию аспекта глагола(?).

\section*{Задание 2.}
\begin{enumerate}
\item \textit{Второкурсники, которым лектор показал фокусы, громко аплодировали.}\\
Непрямое дополнение (\textsc{io})

\item \textit{Мужчина, лицо которого люди по всему миру видят во сне, оказался Хомским.}\\
"Генитивная" ИГ (\textsc{gen})

\item \textit{Слона, который зашел в посудную лавку, я и не приметил.}\\
Субъект (\textsc{su})

\item \textit{Коле, из-за которого всё и случилось, даже не пришло в голову извиниться.}\\
"Косвенное дополнение" (\textsc{obl})

\item \textit{Статья, которую я написал, повергла меня в ужас.}\\
Прямое дополнение (\textsc{do})

\item \textit{?Хаспельмат, прекрасней которого нет, рассказал нам про Грамбанк.}\\
Объект сравнения (\textsc{ocomp})
\end{enumerate}

\section*{Задание 4*.}
После примеров (1)-(2) это эргативный элайнмент. После примера (3): здесь не так то, что абсолютивом кодируются агенсоподобный участник и пациенсоподобный участник. После примера (4) кажется, что носители выбирают способ кодирования участников в зависимости от времени и аспекта.


\end{document}